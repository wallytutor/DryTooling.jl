\addsymbol{t}{\ensuremath{t}}{Time.}

\addsymbol{x}{\ensuremath{x}}{Cartesian spacial coordinate, also used with subscripts for denoting multiple coordinates when required.}

\addsymbol{u}{\ensuremath{u}}{Velocity. If used with a subscript it corresponds to the component along the associated spacial coordinate.}

\addsymbol{l}{\ensuremath{l}}{(Characteristic) length. If used with a subscript it corresponds to the component along the associated spacial coordinate.}

\addsymbol{R}{\ensuremath{R}}{Radius or ideal gas constant.}

\addsymbol{D}{\ensuremath{D}}{Diameter.}

\addsymbol{P}{\ensuremath{P}}{Area or pressure.}

\addsymbol{A}{\ensuremath{A}}{Area.}

\addsymbol{V}{\ensuremath{V}}{Volume.}

\addsymbol{N}{\ensuremath{N}}{Number of a countable or iterations, steps,...}

\addsymbol{phi}{\ensuremath{\phi}}{Denotes a fraction or an angle. It may appear in mixture laws, computations of porosity, etc. A subscript may indicate multiple substances or identify a medium. It is used as an angle mainly with spherical coordinates.}